
\section*{Введение}
\addcontentsline{toc}{section}{Введение}
Единый государственный экзамен (ЕГЭ)~— централизованно проводимый в Российской
Федерации экзамен в средних учебных заведениях — школах, лицеях и гимназиях,
форма проведения ГИА(Государственный Итоговая Аттестация) по образовательным программам среднего общего образования.
Служит одновременно выпускным экзаменом из школы и вступительным экзаменом в вузы.

Но за время обучения в 10 и 11 классе при подготовке к ЕГЭ школьники сталкиваются с дефицитом заданий по определённым категориям.
Так в конце 2021 года в список заданий ЕГЭ были добавлены новые задания под номером 11 по теме «Графики функций», а в конце 2023 — задание №2 по теме «Вектора», количество которых для прорешивания было очень мало. 
А по теме «Производная и первообразная» банк заданий расходуется при подготовке с невероятной скоростью:
так как это преимущественно графические задания, решение их занимает менее минуты, а их составление вручную занимает несоразмерно много времени.

ЕГЭ является относительно неизменяемым экзаменом, поэтому все материалы, которые уже были выложены в открытый доступ, имеют полные решения, что приводят к списыванию учениками.

При этом существуют задания с вспомогательным чертежом. Чаще всего для целого ряда заданий используется одна и та же иллюстрация, которая не всегда соответствуют условиям задачи, а иногда отвлекает от решения.
Проект «Час ЕГЭ» позволяет решить все эти проблемы.

«Час ЕГЭ» — компьютерный образовательный проект, разрабатываемый при математическом
факультете ВГУ в рамках «OpenSource кластера» и предназначенный для помощи учащимся
старших классов при подготовке к тестовой части единого государственного экзамена.
%%ссылочки на доклады
Задания в «Час ЕГЭ» генерируются случайным образом по специализированным алгоритмам,
называемых шаблонами, каждый из которых
охватывает множество вариантов соответствующей ему задачи. Для
пользователей
предназначены четыре оболочки (режима работы): «Случайное задание», «Тесты на печать»,
«Полный тест» и «Мини-интеграция».
«Час ЕГЭ» является полностью открытым (код находится под лицензией GNU GPL 3.0)
и бесплатным.
В настоящее время в проекте полностью реализованы тесты по математике с кратким
ответом (бывшая «часть В»).~\cite{fipi}
Планируется с течением времени включить в проект тесты по другим предметам школьной
программы.

Первая глава этой работы посвящена обзору вспомогательных функций, которые ускоряют написание шаблонов по теме «Планиметрия». Также приведён алгоритм написания шаблона с чертежом.

Вторая глава представляет решение проблемы отрисовки фигур в трёхмерном пространстве на языке программирования JavaScript; рассказывает о применении объектно-ориентированного программирования для упрощения написания шаблонов с чертежом; затрагивает вопрос об написании программного кода при помощи нейросетей; приводит обзор вспомогательных функций и алгоритм написания шаблона по теме «Стереометрия». 
