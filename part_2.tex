\section{Стереометрия}\label{2sect}
\subsection{Разработка библиотек с помощью Gpt-Chat}

На данный момент в языке JavaScript отсутствуют встроенные средства для изображения трёхмерных фигур. И существует только одна подходящая библиотека \texttt{Three.js}, которая могла бы выполнить проецирование координат фигуры на плоскость с учётом положения наблюдателя. Но при этом для создания любого объекта необходима не только камера, но и сцена с рендерингом, что значительно замедляет работу проекта.

\begin{lstlisting}[caption={Код, необходыимый для отрисовки куба}]
import * as THREE from 'three';

// Создать сцену, камеру и рендер
const scene = new THREE.Scene();
const camera = new THREE.PerspectiveCamera(75, window.innerWidth / window.innerHeight, 0.1, 1000);
camera.position.z = 2;
const renderer = new THREE.WebGLRenderer();
renderer.setSize(window.innerWidth, window.innerHeight);

// Создать геометрию куба
const geometry =new THREE.BoxGeometry(1, 1, 1);

// Создать материал куба
const material = new THREE.MeshBasicMaterial({ color: 0x00ff00 });

// Создать куб
const cube = new THREE.Mesh(geometry, material);

// Добавить куб в сцену
scene.add(cube);

// Отрендерить сцену
renderer.render(scene, camera);

\end{lstlisting}

%Заменить на: увеличивает размер проекта. Про сложность вызова возможно. Обосновать почему не подходит. Возможно приложить пример использования. 

Другие подобные ей библиотеки проводят проецирование на плоскость с поворотом только вокруг осей $Ox$ и $Oz$, что при отрисовке более сложных чертежей является недостаточным.

С учётом прогресса современных технологий встал закономерный вопрос, сможет ли нейросеть ChatGPT 3.5 сгенерировать код проектора. Ниже представлен диалог с нейросетью:

\begin{leftBox}
	write a 2D to 3D projector in JS without using libraries. input:  point = \{x,y,z\}, camera= \{x,y,z, angleX, angleY, angleZ, scale\}
\end{leftBox}

\begin{lstlisting}[numbers=none]
	
	// Define the point in 2D
	const point = { x: 100, y: 50, z: 0 };

	// Define the camera parameters
	const camera = { x: 0, y: 0, z: 0, angleX: 0, angleY: 0, angleZ: 0, scale: 1 };

	// Function to project 2D point to 3D
	function project2Dto3D(point, camera) {
			// Apply rotation around X axis
			const rotatedY = point.y * Math.cos(camera.angleX) - point.z * Math.sin(camera.angleX);
			const rotatedZ = point.y * Math.sin(camera.angleX) + point.z * Math.cos(camera.angleX);

			// Apply rotation around Y axis
			const rotatedX = point.x * Math.cos(camera.angleY) + rotatedZ * Math.sin(camera.angleY);
			const rotatedZ2 = -point.x * Math.sin(camera.angleY) + rotatedZ * Math.cos(camera.angleY);

			// Apply rotation around Z axis
			const rotatedX2 = rotatedX * Math.cos(camera.angleZ) - rotatedY * Math.sin(camera.angleZ);
			const rotatedY2 = rotatedX * Math.sin(camera.angleZ) + rotatedY * Math.cos(camera.angleZ);

			// Apply scaling
			const scaledX = rotatedX2 * camera.scale;
			const scaledY = rotatedY2 * camera.scale;

			// Translate to camera position
			const translatedX = scaledX + camera.x;
			const translatedY = scaledY + camera.y;
			const translatedZ = rotatedZ2 * camera.scale + camera.z;

			return { x: translatedX, y: translatedY, z: translatedZ };
		}

	// Project the point from 2D to 3D
	const projectedPoint = project2Dto3D(point, camera);

	// Output the result
	console.log(»Projected 3D point:», projectedPoint);
	
\end{lstlisting}

\begin{leftBox}
	shorten function project2Dto3D
\end{leftBox}

\begin{lstlisting}[numbers=none]
	
	function project2Dto3D(point, camera) {
			const { x, y, z } = point;
			const { angleX, angleY, angleZ, scale, x: cx, y: cy, z: cz } = camera;

			const cosX = Math.cos(angleX);
			const sinX = Math.sin(angleX);
			const cosY = Math.cos(angleY);
			const sinY = Math.sin(angleY);
			const cosZ = Math.cos(angleZ);
			const sinZ = Math.sin(angleZ);

			const rx = x * cosY * cosZ - y * cosY * sinZ + z * sinY;
			const ry = (x * (sinX * sinY * cosZ + cosX * sinZ) + y * (sinX * sinY * sinZ - cosX * cosZ) + z * sinX * cosY) * scale;
			const rz = (-x * (cosX * sinY * cosZ - sinX * sinZ) - y * (cosX * sinY * sinZ + sinX * cosZ) + z * cosX * cosY) * scale;

			return { x: rx + cx, y: ry + cy, z: rz + cz };
		}
	
\end{lstlisting}


\begin{leftBox}
	add comments in code
\end{leftBox}

\begin{lstlisting}[numbers=none]
	
	function project2Dto3D(point, camera) {
			// Extract point coordinates
			const { x, y, z } = point;
			// Extract camera parameters
			const { angleX, angleY, angleZ, scale, x: cx, y: cy, z: cz } = camera;

			// Precompute trigonometric values for rotation angles
			const cosX = Math.cos(angleX);
			const sinX = Math.sin(angleX);
			const cosY = Math.cos(angleY);
			const sinY = Math.sin(angleY);
			const cosZ = Math.cos(angleZ);
			const sinZ = Math.sin(angleZ);

			// Apply rotation around X axis
			const rx = x * cosY * cosZ - y * cosY * sinZ + z * sinY;
			// Apply rotation around Y axis and scale
			const ry = (x * (sinX * sinY * cosZ + cosX * sinZ) + y * (sinX * sinY * sinZ - cosX * cosZ) + z * sinX * cosY) * scale;
			// Apply rotation around Z axis and scale
			const rz = (-x * (cosX * sinY * cosZ - sinX * sinZ) - y * (cosX * sinY * sinZ + sinX * cosZ) + z * cosX * cosY) * scale;

			// Translate to camera position
			const translatedX = rx + cx;
			const translatedY = ry + cy;
			const translatedZ = rz + cz;

			return { x: translatedX, y: translatedY, z: translatedZ };
		}
	
\end{lstlisting}

За несколько шагов удалось получить корректный, оптимизированный код.

\subsection{Применение объектно-ориентированного программирования для разработки шаблонов}

Банк заданий содержит большое количество разнообразных задач по теме «Стереометрия». Поэтому одной из первостепенных задач было сократить код шаблонов и исключить вычислительные ошибки. Для этого были разработаны классы многогранников, которые содержат в себе длины рёбер, объем, площади оснований, а так же тернарную матрицу связности и канонические координаты вершин.

Матрица может содержать значения: 1, 0, либо специальное значение, указывающие на отображение ребра пунктиром.

Пример канонической матрицы связей:

\begin{lstlisting}[caption = {Каноническая матрица связей для параллелепипеда}]
	[   [1],
	    [0, 1],
	    [1, 0, 1],
	    [0, 0, 0, 1],
	    [1, 0, 0, 0, 1],
	    [0, 1, 0, 0, 0, 1],
	    [0, 0, 1, 0, 1, 0, 1],
	];
\end{lstlisting}

Мы можем опускать конец строки матрицы, если он состоит только из нулей, и главную диагональ матрицы (на ней всегда стоят нули).

\textbf{Определение.} Каноническим положением будем называть такое расположение многогранника, когда его высота, проходящая через центр масс его основания, совпадает с осью аппликат и началом координат делится пополам (Рис.~\ref{fig:fig1}-.~\ref{fig:fig2}).

При таком расположении начало координат можно расположить в центре иллюстрации. Тогда чертёж не будет чрезмерно смещён ни в одну из сторон.
\newpage
\begin{multicols}{2}
	\begin{figure}[H]
		\tdplotsetmaincoords{75}{125}
		\begin{tikzpicture}
			[tdplot_main_coords,
				scale=0.9,
				line/.style={very thick, blue},
				linedash/.style={very thick,dashed, blue},
				diagdash/.style={very thick,dashed, blue},
				axis/.style={->,black, thick},
				axisdash/.style={thick,dashed, black},
				figure/.style={opacity=.2,very thick,fill=blue,},]
	
			\coordinate (O) at (0,0,0);
			\coordinate (A) at (-2,-2,-2);
			\coordinate (B) at (-2,2,-2);
			\coordinate (C) at (2,2,-2);
			\coordinate (D) at (2,-2,-2);
	
			\coordinate (A1) at (-2,-2,2);
			\coordinate (B1) at (-2,2,2);
			\coordinate (C1) at (2,2,2);
			\coordinate (D1) at (2,-2,2);
	
			%draw the axes
			\draw[axis] (O) -- (6,0,0) node[anchor=west]{$x$};
			\draw[axis] (O) -- (0,5,0) node[anchor=west]{$y$};
			\draw[axis] (O) -- (0,0,3) node[anchor=west]{$z$};
			\draw[axisdash] (-5,0,0) -- (O);
			\draw[axisdash] (0,-4,0) -- (O);
			\draw[axisdash] (0,0,-3) -- (O);
	
			%draw the bottom of the cube
			\draw[figure] (A) -- (B) -- (C) -- (D) -- cycle;
	
			%draw the back-right of the cube
			\draw[figure] (A) -- (B) -- (B1) -- (A1) -- cycle;
	
			%draw the back-left of the cube
			\draw[figure] (A) -- (D) -- (D1) -- (A1) -- cycle;
	
			%draw the front-right of the cube
			\draw[figure] (D) -- (C) -- (C1) -- (D1) -- cycle;
	
			%draw the front-left of the cube
			\draw[figure] (B) -- (C) -- (C1) -- (B1) -- cycle;
	
			%draw the top of the cube
			\draw[figure] (A1) -- (B1) -- (C1) -- (D1) -- cycle;
	
			\draw[diagdash] (C) -- (A);
			\draw[diagdash] (B) -- (D);
		\end{tikzpicture}
		\captionof{image}{Каноническое положение для параллелепипеда}
		\label{fig:fig1}
	\end{figure}
		\begin{figure}[H]
		\tdplotsetmaincoords{75}{115}
		\begin{tikzpicture}
			[tdplot_main_coords,
				scale=0.9,
				line/.style={very thick, blue},
				linedash/.style={very thick,dashed, blue},
				diagdash/.style={very thick,dashed, blue},
				axis/.style={->,black, thick},
				axisdash/.style={thick,dashed, black},
				figure/.style={opacity=.2,very thick,fill=blue,},]
	
			\coordinate (O) at (0,0,0);
	
			\coordinate (A) at (3,0,-2);
			\coordinate (B) at (0.9270,2.8531,-2);
			\coordinate (C) at (-2.427,1.7633,-2);
			\coordinate (D) at (-2.4270,-1.7633,-2);
			\coordinate (E) at (0.9270,-2.8531, -2);
	
			\coordinate (A1) at (3,0,2);
			\coordinate (B1) at (0.9270,2.8531,2);
			\coordinate (C1) at (-2.427,1.7633,2);
			\coordinate (D1) at (-2.4270,-1.7633,2);
			\coordinate (E1) at (0.9270,-2.8531, 2);
	
			\coordinate (A2) at (-2.43, 0, -2);
			\coordinate (B2) at (-0.75, -2.31,-2);
			\coordinate (C2) at (1.96, -1.43,-2);
			\coordinate (D2) at (1.96, 1.43,-2);
			\coordinate (E2) at (-0.75, 2.31, -2);
	
			%draw the axes
			\draw[axis] (O) -- (9,0,0) node[anchor=west]{$x$};
			\draw[axis] (O) -- (0,4,0) node[anchor=west]{$y$};
			\draw[axis] (O) -- (0,0,3) node[anchor=west]{$z$};
			\draw[axisdash] (-5,0,0) -- (O);
			\draw[axisdash] (0,-4,0) -- (O);
			\draw[axisdash] (0,0,-3) -- (O);
	
			\draw[figure] (A) -- (B)  -- (C)  -- (D) -- (E) -- cycle;
			\draw[figure] (A1) -- (B1)  -- (C1)  -- (D1) -- (E1) -- cycle;
	
			\draw[figure] (A) -- (B)  -- (B1) -- (A1) -- cycle;
			\draw[figure] (C) -- (B)  -- (B1) -- (C1) -- cycle;
			\draw[figure] (D) -- (C)  -- (C1) -- (D1) -- cycle;
			\draw[figure] (D) -- (E)  -- (E1) -- (D1) -- cycle;
			\draw[figure] (A) -- (E)  -- (E1) -- (A1) -- cycle;
	
			\draw[linedash] (A) -- (A2);
			\draw[linedash] (B) -- (B2);
			\draw[linedash] (C) -- (C2);
			\draw[linedash] (D) -- (D2);
			\draw[linedash] (E) -- (E2);
	
		\end{tikzpicture}
		\captionof{image}{Каноническое положение для правильной пятиугольной призмы}
		\label{fig:fig2}
	\end{figure}
\end{multicols}
	
\subsection{Вспомогательные функции}

\subsubsection{Функции для работы с координатами}

\function{verticesInGivenRange}{vertex, {startX, finishX, startY, \\finishY}}
Возвращает \texttt{true}, если двумерная координата точки \texttt{vertex} вида \texttt{\{x,y\}} находится в некоторой прямоугольной области, иначе \texttt{false}.

\function{autoScale}{vertex3D, camera, vertex2D, {startX, finishX, startY, finishY, step, maxScale}}
Увеличивает свойство объекта \texttt{camera.scale} до тех пор, пока все двумерные координаты \texttt{vertex2D} вида \texttt{\{x,y\}}  находится в некоторой прямоугольной области. \texttt{step} по умолчанию $0.1$.

\function{distanceFromPointToSegment}{point, segmentStart, segmentEnd}
Возвращает длину перпендикуляра между двумерной точкой \texttt{point} вида \texttt{\{x,y\}} и прямой, проходящей через точки \texttt{segmentStart} и \texttt{segmentEnd}.

\subsubsection{Функции для работы с canvas}

\prototype[\\vertex, matrixConnections]{CanvasRenderingContext2D}{drawFigure}
Соединяет линиями точки массива \texttt{vertex} с элементами \texttt{\{x,y\}} в соответствии с матрицей связей \texttt{matrixConnections}, которая является массивом, содержащим в себе 0, 1 или массив step, указывающий на отрисовку пунктиром.

Пример матрицы связей:
\begin{lstlisting}[numbers=none]
	let matrixConnections = [
			[1],
			[strok, strok],
			[0, 0, strok],
			[1, 0, 0, 1],
			[0, 1, 0, 1, 1]
		];
	\end{lstlisting}

\prototype[\\vertex, matrixConnections]{CanvasRenderingContext2D}{drawFigureVer2}
Соединяет линиями точки массива \texttt{vertex} с элементами \texttt{\{x,y\}} в соответствии с матрицей связей \texttt{matrixConnections}. Эта матрица представляет собой объект, где каждое числовое поле соответствует номеру вершины в массиве \texttt{vertex}. В каждом поле находится массив номеров других вершин, с которыми должна быть соединена данная вершина.

Пример матрицы связей:
\begin{lstlisting}[numbers=none]
	let matrixConnections = {
		0: [1, [3, stroke], 5],
		2: [1, [3, stroke], 7],
		4: [[3, stroke], 5, 7],
		9: [1, 8, 10],
		11: [8, 10, 12],
		13: [5, 8, 12],
		15: [7, 10, 12],
	};
	\end{lstlisting}

\subsection{Этапы разработки шаблоны с вспомогательным чертежом по теме «Стереометрия»}

Для примера возьмём задание №27074~\cite{sdamgia}.

\includegraphics*[width= 0.8\linewidth]{2774}

Заготовка шаблона имеет вид:

\lstinputlisting[]{code/27074_1.js}

\begin{enumerate}
	\item Создадим объект класса \texttt{Parallelepiped} со случайной высотой, шириной и глубиной в заданном диапазоне. 
	\lstinputlisting[]{code/27074_2.js}
	\item Определим переменную \texttt{camera}, которая будет отвечать за положение наблюдателя. Спроецируем канонические координаты параллелепипеда на двумерную плоскость при помощи функции \texttt{project3DTo2D}, отмасштабируем полученные координаты так, чтобы они занимали максимально заполняли иллюстрацию, функцией \texttt{autoScale}.
	\lstinputlisting[]{code/27074_3.js}
	\item Перемещаемся в середину иллюстрации. Отрисовываем фигуру функцией \texttt{drawFigure}, передав в неё матрицу связей для параллелепипеда. 
	\lstinputlisting[]{code/27074_4.js}
	\item Далее вырезаем из условия значения и заменяем их данными из класса. Впишем ответ. Обособляем имена фигур в \$\dots\$. Добавляем буквы на вершины параллелепипеда. Добавляем модификаторы  \\        \texttt{NAtask.modifiers.assertSaneDecimals()} (исключает нецелый ответ) и
	\texttt{NAtask.modifiers.variativeABC(letter)} (заменяет все буквы в задании на случайные).
	\lstinputlisting[]{code/27074_5.js}
\end{enumerate}
