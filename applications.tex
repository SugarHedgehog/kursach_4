\section*{Приложение}
\addcontentsline{toc}{section}{Приложение}

\setcounter{lstlisting}{1}

\subsection*{ Шаблоны по теме Планиметрия}
\addcontentsline{toc}{subsection}{ Шаблоны по теме Планиметрия}

\lstinputlisting[caption = 105.js]{code/105.js}
\subsubsection*{Примеры генерируемых задач 105.js}   

\task{В треугольнике  $VAF$ $VA=VF$, $AF=75$, высота $AU$ равна $39$. Найдите $\sin \angle VFA$.}{0,52}{555103857093072n0}
\task{В треугольнике  $NCP$ $CP=CN$, $PN=40$, высота $VP$ равна $6$. Найдите $\sin \angle CNP$.}{0,15}{775585497504985n0}
\task{В треугольнике  $PKT$ $KT=PK$, $TP=4$, высота $TY$ равна $2$. Найдите $\sin \angle KPT$.}{0,5}{6338310394113642n0}

\lstinputlisting[caption = 2069.js]{code/2069.js}
\subsubsection*{Примеры генерируемых задач 2069.js}   

\task{Угол между биссектрисой и медианой прямоугольного треугольника, проведёнными из вершины прямого угла, равен $38^{\circ}$. Найдите больший угол прямоугольного треугольника. Ответ дайте в градусах.}{83}{975621288543568n0}
\task{Угол между биссектрисой и медианой прямоугольного треугольника, проведёнными из вершины прямого угла, равен $8^{\circ}$. Найдите меньший угол прямоугольного треугольника. Ответ дайте в градусах.}{37}{552764015073117n0}
\task{Угол между биссектрисой и медианой прямоугольного треугольника, проведёнными из вершины прямого угла, равен $31^{\circ}$. Найдите меньший угол прямоугольного треугольника. Ответ дайте в градусах.}{14}{673204479916529n0}

\lstinputlisting[caption = 27764.js]{code/27764.js}
\subsubsection*{Примеры генерируемых задач 27764.js}   

\task{В треугольнике $RHC$ угол $R$ равен $67^{\circ}$, углы $H$ и $C$ – острые, биссектрисы $LH$ и $SC$ пересекаются в точке $X$. Найдите угол $CXH$. Ответ дайте в градусах.}{123,5}{7108522192090796n0}
\task{В треугольнике $YHA$ угол $Y$ равен $45^{\circ}$, углы $H$ и $A$ – острые, биссектрисы $HW$ и $AR$ пересекаются в точке $M$. Найдите угол $AMH$. Ответ дайте в градусах.}{112,5}{6232978139752687n0}
\task{В треугольнике $URF$ угол $U$ равен $7^{\circ}$, углы $R$ и $F$ – острые, биссектрисы $AR$ и $FQ$ пересекаются в точке $X$. Найдите угол $FXR$. Ответ дайте в градусах.}{93,5}{360610115591288n0}

\newpage
\subsection*{ Шаблоны по теме Стереометрия}   
\addcontentsline{toc}{subsection}{ Шаблоны по теме Стереометрия}   

\lstinputlisting[caption = 3011.js]{code/3011.js}
\subsubsection*{Примеры генерируемых задач 3011.js}   

\task{В правильной четырёхугольной пирамиде $OQSDX$ с основанием $QSDX$ боковое ребро равно $\sqrt{1489,5}$, сторона основания равна $39$. Найдите объём пирамиды.}{13689}{314379431351734n0}
\task{В правильной четырёхугольной пирамиде $VEBXI$ с основанием $EBXI$ боковое ребро равно $\sqrt{848,5}$, сторона основания равна $27$. Найдите объём пирамиды.}{5346}{438948036652505n0}
\task{В правильной четырёхугольной пирамиде $WYFDC$ с основанием $YFDC$ боковое ребро равно $\sqrt{1513}$, сторона основания равна $24$. Найдите объём пирамиды.}{6720}{087863431801837n0}

\lstinputlisting[caption = 27114.js]{code/27114.js}
\subsubsection*{Примеры генерируемых задач 27114.js}

\task{Объём правильной четырёхугольной пирамиды $UCXOZ$ равен $31164$. Точка $Y$ – середина ребра $UC$. Найдите объём треугольной пирамиды $YCXZ$.}{7791}{455385802312436n0}
\task{Объём правильной четырёхугольной пирамиды $TKUWG$ равен $4800$. Точка $L$ – середина ребра $TK$. Найдите объём треугольной пирамиды $LKUG$.}{1200}{165146394950057n0}
\task{Объём правильной четырёхугольной пирамиды $UOVSZ$ равен $25650$. Точка $B$ – середина ребра $UO$. Найдите объём треугольной пирамиды $BOVZ$.}{6412,5}{5369517169326834n0}

\lstinputlisting[caption = 27115.js]{code/27115.js}

\subsubsection*{Примеры генерируемых задач 27115.js}
\task{Объём треугольной пирамиды равен $ 2560 $. Через вершину пирамиды и среднюю линию её основания проведена плоскость (см. рисунок). Найдите объём отсечённой треугольной пирамиды.}{640}{905430173181577n0}
\task{Объём треугольной пирамиды равен $ 4950 $. Через вершину пирамиды и среднюю линию её основания проведена плоскость (см. рисунок). Найдите объём отсечённой треугольной пирамиды.}{1237,5}{378093624649185n0}
\task{Объём треугольной пирамиды равен $ 12096 $. Через вершину пирамиды и среднюю линию её основания проведена плоскость (см. рисунок). Найдите объём отсечённой треугольной пирамиды.}{3024}{224079401047001n0}

\lstinputlisting[caption = 12.js]{code/12.js}

\subsubsection*{Примеры генерируемых задач 12.js}

\task{Найдите объём многогранника, изображённого на рисунке (все двугранные углы – прямые).}{3402}{698073271409958n0}
\task{Найдите объём многогранника, изображённого на рисунке (все двугранные углы – прямые).}{3156}{668210454498093n0}
\task{Найдите объём многогранника, изображённого на рисунке (все двугранные углы – прямые).}{4265}{193153928573128n0}

\lstinputlisting[caption = 29193.js]{code/27193.js}

\subsubsection*{Примеры генерируемых задач 27193.js}

\task{Найдите площадь поверхности многогранника, изображённого на рисунке (все двугранные углы – прямые).}{948}{020701664257409n0}
\task{Найдите объём многогранника, изображённого на рисунке (все двугранные углы – прямые).}{2871}{636805691610557n0}
\task{Найдите площадь поверхности многогранника, изображённого на рисунке (все двугранные углы – прямые).}{1454}{392642674576018n0}

\lstinputlisting[caption = 144526144.js]{code/144526144.js}

\subsubsection*{Примеры генерируемых задач 144526144.js}

\task{Найдите площадь поверхности многогранника, изображённого на рисунке.}{3384}{3067148411583609n0}
\task{Найдите площадь поверхности многогранника, изображённого на рисунке.}{2656}{8011054822883317n0}
\task{Найдите площадь поверхности многогранника, изображённого на рисунке.}{2633}{4457923588593404n0}
