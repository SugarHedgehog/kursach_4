\documentclass[a4paper, 12pt]{extarticle}
\usepackage{fontspec}
\usepackage{polyglossia}
\setmainfont{CMU Serif}
\newfontfamily{\cyrillicfont}{CMU Serif}
\setsansfont{CMU Sans Serif}
\newfontfamily{\cyrillicfontsf}{CMU Sans Serif}
\setmonofont{CMU Typewriter Text}
\newfontfamily{\cyrillicfonttt}{CMU Typewriter Text}
\setdefaultlanguage{russian}
\usepackage{amsfonts}

\usepackage[left=1.5cm,right=1cm,top=2cm,bottom=2cm]{geometry}
\title{Автоматическая генерация фонда оценочных средств ЕГЭ по математике по теме "Стереометрия"}
\author{Суматохина Александра 4 курс Кафедра Теории функции и геометрии\\Научный руководитель: Авдеев Н.Н.}
\date{23 апреля 2024}
\begin{document}
\maketitle

Здравствуйте, меня зовут Суматохина Александра, я обучаюсь на 4 курсе, мой научный руководитель Семёнов Евгений Михайлович, научный консультант Авдеев Николай Николаевич. Тема моего доклада автоматическая генерация фонда оценочных средств ЕГЭ по математике по теме "Стереометрия".

\subsection*{Существующие проблемы}
Но за 10 и 11 класс при подготовке к ЕГЭ школьники сталкиваются с дефицитом заданий по определённым категориям.
Так в конце 2021 года в список заданий ЕГЭ были добавлены новые задания под номером 11 по теме "графики функции", а в конце 2023 - задание номер 2 по теме "вектора", количество которых, для прорешивания было очень мало. 
А по теме "Производная и первообразная" банк заданий с невероятной скоростью.

Так как это преимущественно графические задания, решение их занимает менее минуты, а их составление вручную занимает несоразмерно много времени.

ЕГЭ является относительно неизменяемым экзаменом, поэтому все материалы, которые уже были выложены в открытый доступ имеют полные решения, что приводят к списыванию учениками.

При этом существуют задания с вспомогательным чертежом. Чаще всего для целого ряда заданий используется одна и та же иллюстрация, которая не всегда соответствуют условиям задачи, а иногда отвлекают от решения.
Проект «Час ЕГЭ» позволяет решить все эти проблемы.

Первая глава этой работы посвящена обзору вспомогательных функций, которые ускоряют написание шаблонов по теме «Планиметрия». Также приведён алгоритм написания шаблона с чертежом.

Вторая глава представляет решение проблемы отрисовки фигур в трёхмерном пространстве на языке программирования JavaScript; рассказывает о применении объектно-ориентированного программирования для упрощения написания шаблонов с чертежом; затрагивает вопрос об написании программного кода при помощи нейросетей; приводит обзор вспомогательных функций и алгоритм написания шаблона по теме «Стереометрия». 


\subsection*{Проект «Час ЕГЭ»}
«Час ЕГЭ» — компьютерный образовательный проект, разрабатываемый при математическом
факультете ВГУ в рамках «OpenSource кластера» и предназначенный для помощи учащимся
старших классов, учителями и репетиторам при подготовке к тестовой части единого государственного экзамена.

Задания в «Час ЕГЭ» генерируются случайным образом по специализированным алгоритмам, называемых шаблонами, каждый из которых охватывает множество вариантов соответствующей ему задачи.

\subsection*{Достижения}
\begin{itemize}
	\item Полностью покрыт банк заданий ФИПИ по теме «Планиметрия»
\end{itemize}
    В ядро добавлены функции для отрисовки:
    \begin{itemize}
        \item Условных обозначений на чертежах, такие как штрих-метка
		\item Углов, в отдельности прямых
		\item Обозначений для равных углов
		\item Отрезков заданной длины под некоторым углом
		\item Строк на середине отрезка
    \end{itemize}
	
\subsection*{Проблема отрисовки многогранников в JavaScript}

\begin{itemize}
	\item Отсутствуют встроенные средства для изображения трёхмерных фигур
	\item На данный момент существует только одна подходящая библиотека \texttt{Three.js}, которая могла бы выполнить проецирование координат фигуры на плоскость с учётом положения наблюдателя.
	\item При для создания любого объекта необходима не только камера, но и сцена, рендеринг и материал фигуры, что значительно замедляет работу проекта.
	\item Подобные ей проводят проецирование на плоскость с поворотом только вокруг осей $OX$ и $OZ$
	\item Ранне уже попытки написания функции проекций фигур. Но чертежи не соответствовали условиям задач.
\end{itemize}

С учётом прогресса современных технологий, встал закономерный вопрос, сможет ли нейросеть ChatGPT 3.5 сгенерировать код проектора.
За несколько шагов удалось получить корректный, оптимизированный код.

\subsection*{Сокращение кода и введение канонических координат}

Стоит отметить, что задач по теме "Стереометрия" огромное множество. Поэтому одной из первостепенных задач было сократить код шаблонов и исключить вычислительные ошибки. Для этого были разработаны классы многогранников, которые содержат в себе длины рёбер, объем, площади оснований, а так же тернарную матрицу связности и канонические координаты вершин.

Матрица может содержать значения: 1, 0, либо специальное значение, указывающий на отображении ребра пунктиром.

Каноническим положением будем называть такое расположение многогранника, когда его высота, проходящая через центр масс его основания, совпадает с осью аппликат и начало координат делится пополам.

При таком расположении, начало координат можно расположить в центре иллюстрация. Тогда чертёж не будет смещён ни в одну из сторон.

Так написание шаблона было сведено к 5 пунктам. 

\subsection*{Этапы генерации}
	\begin{itemize}
		\item Создание объекта нужного класса (фигуры)
		\item Преобразование канонических координат на двумерную плоскость при помощи функции \texttt{project3DTo2D}
		\item Масштабирование координат функцией \texttt{autoScale}
		\item Корректирование матрицы связей (добавление диагоналей или сечений)
		\item Отрисовка фигуры \texttt{drawFigure}
	\end{itemize}

\subsection*{Достижения}
\begin{itemize}
	\item Полностью покрыт открытый банк заданий ФИПИ по темам: параллелепипеды, призмы, кубы.
	\item Разработано 35 шаблонов
	\item Проведён эксперимент по написанию проектора из $\mathbb{R}^3 \to \mathbb{R}^2$ с помощью ChatGPT 3.5 на языке программирования JavaScript.
	\item Написана функция отрисовки фигуры на основе её координат и матрицы связности вершин
	\item Написана функция автомасштабирования фигуры.
\end{itemize}

\subsection*{Итоги}
Как итог, могу сказать, что нейросети способны генерировать краткий, корректный и оптимизированный код код с комментариями и тестами к нему. Что существенно увеличивает продуктивности программиста.

\end{document}